\subsection{Jacobi's method}

\begin{align*}
t &= \tau \pm \sqrt{1+\tau^2}\\ 
c &= \cos \theta = \frac{1}{\sqrt{1+\tan^2\theta}} = \frac{1}{\sqrt{1+t^2}}\\
s &= \sin \theta = \tan \theta \cos \theta = t c\\
\end{align*}

\subsection{The algorithm}



	Metode
Algorigmer: Jacobi, rhomax
valg av epsilon ("0")

Forelesning: Konvergens etter mellom 12n3 og 20n3, vi har mye kjappere!!!!! - fordi sym!!!


Our matrix \textbf{A} is symmetric, meaning that $ \textbf{A}^T = \textbf{A} $. From this it follows that the product $ \textbf{S}^TAS $ also is symmetric, as is shown under:

\begin{align}
\textbf{S}^T\textbf{AS} &= \textbf{S}^T\textbf{A}^T(\textbf{S}^T)^T\\
&= \textbf{S}^T\textbf{AS} 
\end{align}

Thus is it sufficient to search only the elements above the diagonal for the highest valued element in the matrix, instead of the entire matrix. 



\subsection{Unit tests}

In this project we used two unit tests to ensure that the program performs as expected and delivers accurate enough results. 

One way to be sure that our algorithm gives correct answers is to task it to find the eigenvalues of a matrix with known eigenvalues. These values had been pre calculated by Matlab. 

The other unit test we utilized was to check that our algorithm to find the largest off-diagonal elements actually found the largest off-diagonal elements. This was done by setting up a known matrix and tasking our algorithm to find the largest off-diagonal element and checking it against the manually found largest element. 




Other tests: Orthogonality, norm(A) = norm(B), symmetric


\textbf{HVORFOR SØKER VI BARE ØVRE HALVDEL???????????????????????????????????????}