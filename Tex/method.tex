\subsection{Jacobi's method}

\begin{align*}
t &= \tau \pm \sqrt{1+\tau^2}\\ 
c &= \cos \theta = \frac{1}{\sqrt{1+\tan^2\theta}} = \frac{1}{\sqrt{1+t^2}}\\
s &= \sin \theta = \tan \theta \cos \theta = t c\\
\end{align*}

\subsection{The Jacobi algorithm}

The essence of the algorithm is to continue to do similarity transforms on the matrix A until it is more or less diagonalized. Because the off-diagonal elements will approach $ 0 $, it is sufficient to count until the largest off-diagonal element is $ < 10^{-10} $. 

Our matrix \textbf{A} is symmetric, which means that $ \textbf{A}^T = \textbf{A} $. From this it follows that the product $ \textbf{S}^TAS $ also is symmetric, as is shown under:

\begin{align}
\textbf{S}^T\textbf{AS} &= \textbf{S}^T\textbf{A}^T(\textbf{S}^T)^T\\
&= \textbf{S}^T\textbf{AS} 
\end{align}

Thus  it is sufficient to search only the elements above the diagonal for the highest valued element in the matrix, instead of the entire matrix. This greatly speeds up the searching-algorithm, as you half the number of elements to read in order to find the greatest value. 



Velger for hvilken $ \tau $ har $ \mp $?????
Hvorfor deler på?????
OBS: FEIL I TEORI : skal være -tau!!!



\begin{align}
t &= \frac{(-\tau \pm \sqrt{1+\tau^2})(-\tau \mp \sqrt{1+ \tau^2})}{-\tau \mp \sqrt{1+\tau^2}}\\
&= \frac{-1}{-\tau \mp \sqrt{1+\tau^2}}
\end{align}

	Metode
Algorigmer: Jacobi, rhomax
valg av epsilon ("0")








\subsection{Unit tests}

In this project we used two unit tests to ensure that the program performs as expected and delivers accurate enough results. 

One way to be sure that our algorithm gives correct answers is to task it to find the eigenvalues of a matrix with known eigenvalues. These values had been pre calculated by Matlab. 

The other unit test we utilized was to check that our algorithm to find the largest off-diagonal elements actually found the largest off-diagonal elements. This was done by setting up a known matrix and tasking our algorithm to find the largest off-diagonal element and checking it against the manually found largest element. 





