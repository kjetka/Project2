
\begin{table}[H]\caption{Convergence of eigenvalues for the non-interacting case using the Jacobi method. The energies corresponding to the eigenvalues $\lambda_1$, $\lambda_2$ and $\lambda_3$ are ?? eV, ?? eV and ?? eV respectively.}
	\label{tab:eigval}
	\begin{tabular}{cccc}
		N   &   $\lambda_1$  &  $\lambda_2$  &  $\lambda_3$    
\hline
10     &          2.9263   &    6.61967   &    10.0351
100     &          2.99928   &    6.99642   &    10.9919
200     &          2.99982   &    6.99911   &    10.9986
400     &          2.99996   &    6.99979   &    11.0003

	\end{tabular}
\end{table}


\begin{table}[H]\caption{Comparisson of the time used between the Jacobi-algorithm and the Armadillo function  $eig\_sym$ as a function of the mesh size N of matrix A. "Transforms" refers to the amount of similarity transforms the Jacobi method uses.}
	\label{tab:time}
	\begin{tabular}{cccccc}
		N       &   # Similarity transforms   &  time Jacobi (s)      & time Armadillo (ms)            
\hline
10      &   144      &   5.7e-05      &   0.04
100      &   17681      &   0.211299      &   1.391
200      &   71500      &   2.83662      &   7.171
400      &   287490      &   58.0194      &   22.261

	\end{tabular}
\end{table}




\begin{table}[H]\caption{Eigenvalues for the interacting case. All the calculations are done with a mesh size of N=400. $ \rho_{max} $ was set to $ 7.9 $ for all $ \omega_r $, except for $ \omega_r  = 0.01$ ( $ \rho_{max} =50$). }
	\label{tab:omega}
	\begin{tabular}{cccc}
		\input{../Outfiles/results_omega.txt}
	\end{tabular}
\end{table}

Forelesning: Konvergens etter mellom 12n3 og 20n3, vi har mye kjappere!!!!! - fordi sym!!!
$ \omega_r= 0.25 $ gir 1/2 verdien av egenverdien gitt for tilfellet n=2 i \cite{litterature}. 