\subsection{The boundaries - $\rho_{max}$}

Because we used the radial Schrödinger Equation our boundary conditions are given by $r\in[0,\infty)$ and since $\rho = r/\alpha$ the same boundary conditions are still relevant. That means that $\rho_{max}$ should be $\infty$ that is not possible however on a computer.

Therefore an approximation of the boundary conditions had to be made. We chose $\rho_{max} = ?? $ for all our calculations except when $\omega_r = 0.01$ then we changed it to be $\rho_{max} = 50$. The small frequency made the wave function smear out, and we needed a bigger interval to make sure we had the whole function in our calculations. 

We also noticed that the result was very dependent on $\rho_{max}$ especially if we chose a too small value. We used our knowledge of the analytical result to find a good $\rho_{max}$, but without this knowledge it would have been difficult. It is important to make sure that the boundary conditions are reflecting the reality in a good way to get the right results. 

\subsection{Efficiency of method}

As we can see in Table \ref{tab:time} the CPU time of our Jacobi's method is big compared to armadillo's eigenvalue solver function. We needed 400 mesh points to get a good accuracy for the eigenvalues and if we would have needed more it would have been a very time consuming job.

The number of similarity transformations was big and it give a picture of why the CPU time is big. 

This might indicate that 

Discuss the CPU time and the convergence. Not a perfect method?

Discuss iterations (symmetric matrix + tridiagonal).

Choose of tolerance.

Specialize the algorithm to a tridiagonal matrix?

\subsection{Unit test. Kjetil}

Good unit test? Better/other unit tests?

